\documentclass[10pt]{article}

\pdfoutput=1 
\def\baselinestretch{1.06} 

%\usepackage[math]{iwona}
\usepackage[T1]{fontenc}

\usepackage{amsmath,amssymb,amsfonts,mathrsfs,amsthm}
%\usepackage{cmbright}
\usepackage[inline]{showlabels} 
\usepackage[]{graphicx}
\usepackage{xcolor,dsfont}
\definecolor{darkblue}{rgb}{0.1,0.1,.7}
\usepackage[colorlinks, linkcolor=darkblue, citecolor=darkblue, urlcolor=darkblue, linktocpage]{hyperref} 
\usepackage[square, comma, compress,numbers]{natbib}
\usepackage{geometry}
\geometry{letterpaper,tmargin=3.0cm,bmargin=3.0cm,lmargin=3.0cm,rmargin=3.0cm}
\usepackage[margin=10pt,font=small,labelfont=bf]{caption}
%\numberwithin{equation}{section}

\usepackage{youngtab}

%%%%%%%%%%%%% specific commands

\theoremstyle{plain}
\newtheorem*{lem}{Lemma}
\newtheorem{lemnum}{Lemma}[section]
\newtheorem*{fact}{Fact}
\theoremstyle{remark}
\newtheorem*{rem}{Remark}

%\def\zcc{\mathbb{Z}_2^{\mathcal{C}}}
\def\zcc{\mathcal{C}}
\def\vol{\text{vol}(S^{d-1})}
\def\bdelta{\widetilde{\boldsymbol{a}}}
\def\bla{\boldsymbol{\lambda}}
\def\bxi{\boldsymbol{\xi}}
\def\Ricci{\mca{R}}
\def\sU{\mathrm{U}}
\def\ssU{\mathrm{SU}}
\def\NN{\mathrm{N}}
\def\SS{\mathrm{S}}
\def\be{\mathbf{e}}
\def\bm{\mathbf{m}}
\def\lmax{{\l_\text{max}}}
%% \def\vol{\text{vol}}
\def\qaq{\quad \text{and} \quad}
\def\qor{\quad \text{or} \quad}
\def\dag{\dagger}
\def\phs{{\phantom{*}}}
\def\phd{{\phantom{\dagger}}}
\def\wh{\widehat}
\def\vphi{\varphi}
\def\wt{\widetilde}
\def\bzeta{\boldsymbol{\zeta}}
\def\lra{\leftrightarrow}
\def\La{\Lambda}
\def\VV{\mathcal{V}}
\def\SO{\mathrm{SO}}
\def\Sd{\mathrm{S}_d}

\newcommand{\sset}[2]{\left\{ \, #1 \mid #2 \, \right\}}
\newcommand{\NO}[1]{{:\!#1\!:}}
\newcommand{\norm}[1]{\lvert #1 \rvert}
\newcommand{\ud}[2]{^{#1}_{\phantom{#1}#2}}
\newcommand{\du}[2]{_{#1}^{\phantom{#1}#2}}

\newcommand{\Orange}{\color [rgb]{1,.5,0}}
\newcommand{\MH}[1]{{\Orange\bf [MH: #1]}}  

%% QM commands
\newcommand{\ceil}[1]{\left \lceil #1 \right \rceil }
\newcommand{\floor}[1]{\left \lfloor #1 \right \rfloor }
\newcommand{\braket}[3]{\langle #1|#2|#3 \rangle}
\newcommand{\brakket}[2]{\langle #1|#2\rangle}
\newcommand{\ket}[1]{|#1\rangle}
\newcommand{\bra}[1]{\langle #1|}
\newcommand{\expec}[1]{\langle #1 \rangle}

\def\dps{\displaystyle}
\def\ldef{\mathrel{\mathop:}=}
\def\rdef{=\mathrel{\mathop:}}
\newcommand{\limu}[1]{\mathrel{\mathop{\sim}\limits_{\scriptstyle{#1}}}}

%% structural TeX commands
\def\fns{\footnotesize}
\newcommand{\reef}[1]{(\ref{#1})}
\def\beq{\begin{equation}} 
\def\eeq{\end{equation}}
\def\nn{\nonumber} 
\def\bsub{\begin{subequations}}
\def\esub{\end{subequations}}

%% math styles
\def\mbb{\mathbb}
\def\mbf{\mathbf}
\def\mca{\mathcal}
\def\mfr{\mathfrak}
\def\mrm{\mathrm}
\def\msc{\mathscr}
\def\mtt{\mathtt}
\def\msf{\mathsf}

%% math symbols
% \def\vee{\, \mathrm{v}\, }
\def\th{\tfrac{1}{2}}
\def\half{\frac{1}{2}}
\def\lab{\bar{\lambda}}
\def\pd{\partial}
\def\a{\alpha}
\def\b{\beta}
\def\g{\gamma}
\def\dd{\delta}
\def\ka{\kappa}
\def\la{\lambda}
\def\ga{\gamma}
\def\ze{\zeta}
\def\DD{\Delta}
\def\Oo{\mathcal{O}}
\def\sO{\mathrm{O}}
\def\l{\ell} 
\def\eps{\epsilon}
\def\bz{\boldsymbol{\zeta}}
\def\vareps{\varepsilon}
\def\bn{\mathbf{n}}
\def\unit{\mathds{1}} % needs dsfont package

\def\ddn{\mathsf{d}_N}
\def\OON{\mathrm{O}(N)}
\def\ba{\boldsymbol{a}}
\def\bla{\boldsymbol{\lambda}}
\newcommand{\threej}[6]{ \begin{pmatrix} #1 & #2 & #3 \\ #4 & #5 & #6 \end{pmatrix}}

\def\Red{\color [rgb]{0.9,0.1,0.1}}
\def\Green{\color [rgb]{0.3,0.5,0.2}}
\def\Blue{\color [rgb]{0.3,0.5,0.8}}
\def\dim{\text{dim}}

\setlength{\parskip}{0.1in}
\hyphenpenalty=1000

\begin{document}

\title{A scalar field on the strip}
\author{MJH}
\date\today

\maketitle

Consider a scalar field with mass $m_0^2$ on $[0,L]$, which boundary conditions $\phi(0) = \phi(L) = 0$. The appropriately normalized wavefunctions are
\beq
f_n(x) = \frac{1}{\sqrt{ \omega_n L}} \, \sin\!\left(\frac{n\pi x}{L}\right)\!,
\quad
\omega_n = \sqrt{ \left( \frac{n \pi}{L} \right)^2 + m_0^2}
\eeq
for $n=1,2,\ldots$. Hence
\beq
\phi(t=0,x) = \sum_{n \in \mbb{N}} f_n(x) \left[ a_n^\phd + a_n^\dagger \right].
\eeq
As a consistency check, we have
\beq
\pi(t=0,x) = \frac{\pd}{\pd t} \phi(t,x) \Big|_{t=0} = - i  \sum_{n \in \mbb{N}} \omega_n f_n(x) \left[ a_n^\phd - a_n^\dagger \right]
\eeq
so
\beq
[\phi(0,x),\pi(0,x')] = \frac{2i}{L} \sum_{n \in \mbb{N}}  \sin\!\left(\frac{n \pi x}{L}\right)\sin\!\left(\frac{n \pi x'}{L}\right) = i \delta(x-x').
\eeq

The interaction we consider is
\beq
H = H_0 + g V,
\quad
V =  \int_0^L\!dx\, \NO{\phi^2(t=0,x)}.
\eeq
This corresponds to shifting the mass as $m_0^2 \mapsto m_0^2 + 2g$.  We expect that the gap between the vacuum and the first excited state becomes
\beq
\sqrt{ \left( \frac{\pi}{L} \right)^2 + m_0^2 + 2g} = \omega_1 + \frac{g}{\omega_1} - \frac{g^2}{2\omega_1^3} + \sO(g^3).
\eeq
We can also predict the shift of the vacuum energy. The propagator in Euclidean time is
\beq
G_E(\tau,x|0,x') = \sum_n f_n(x) f_n(x') e^{-\omega_n \tau}.
\eeq
Hence
\beq
\expec{V(\tau)V(0)} = 2 \int_0^L\!dx\, dx' G_E(\tau,x|0,x')^2 = \sum_{n} \frac{1}{2\omega_n^2} e^{-2\omega_n \tau}.
\eeq
and the Casimir energy is given by
\beq
E_\text{vac}(g) = 0 - g^2 \int_0^\infty\!d\tau\,  \expec{V(\tau)V(0)}  + \sO(g^3) = - \frac{g^2}{4} \sum_n \frac{1}{\omega_n^3} + \sO(g^3).
\eeq

\beq
V = \sum_{m,n} A_{mn} \left[ a_m^\dagger a_n^\dagger + 2 a_m^\dagger a_n^\phd + a_m^\phd a_n^\phd \right]
\eeq
where
\[
A_{mn} =  \int_0^L\!dx\, f_m(x) f_n(x) = \frac{\delta_{mn}}{2\omega_m}.
\]
At second order in perturbation theory, what is the spectral density $\rho_0(E)$ of the vacuum state $\ket{0}$ and likewise $\rho_1(E)$ of the first excited state $\ket{1}$?
In formulas 
\beq
\rho_i(E) \ldef \sum_{\ket{\psi} \neq \ket{i}} \delta(E-E_i) |\braket{i}{V}{\psi}|^2.
\eeq
For example,
\beq
V\ket{0} = \sum_{n>m} 2 A_{mn} \ket{m,n} + \sum_{n \geq 1} \sqrt{2} A_{nn} \ket{n,n}.
\eeq
%  \sum_{n=1}^\infty \frac{1}{\sqrt{2} \omega_n} \ket{n,n},
%\quad
%\ket{n,n} = \frac{(a_n^\dagger)^2}{\sqrt{2}} \ket{0}
and
\begin{multline}
V\ket{1} = (\ldots) \ket{1} + \sum_{n \geq 2 } 2 A_{1n} \ket{n} + \sqrt{6} \ket{1,1,1} + \sum_{n \geq 2} \sqrt{2} A_{nn} \ket{1,n,n} \\ + \sum_n 2\sqrt{2} A_{1,n} \ket{1,1,n} + \sum_{n > m \geq 2} 2 A_{mn} \ket{1,m,n}.
\end{multline}
\end{document}



